\documentclass{article}
\usepackage{graphicx} % Required for inserting images
\usepackage{sectsty} % Required for section modification
\usepackage[normalem]{ulem} % Required for underlining
\usepackage{lipsum} % For generating dummy text
\usepackage{geometry} % To modify the page layout

% Page layout
\geometry{a4paper, margin=1in}

\title{\textbf{\underline{Compte rendu Informatique}}}
\author{Samuel SILVA,Yanis ZIDI,Yssam BAIROUKI,Nathan WOHL, Hicham TOUGRI}
\date{30 mai 2023}

\begin{document}

\maketitle

\begin{abstract} % Start the abstract
Le but de ce projet était de développer "Computer Academy", un site Internet dédié à l'apprentissage de la programmation, à la manière d'OpenClassrooms. Le site offre aux utilisateurs la possibilité d'ajouter, de visualiser, et de supprimer des cours sur différents langages de programmation.
\end{abstract}

\sectionfont{\underline} % Pour souligner les titres de sections
\subsectionfont{\normalfont} % Pour ne pas mettre en gras les sous-sections
\section{Technologies Utilisées}

Pour réaliser ce projet, j'ai utilisé les technologies suivantes :

\begin{itemize}
\item \textbf{HTML/CSS} : pour la structure et la présentation du site
\item \textbf{JavaScript} : pour ajouter de l'interactivité au site
\item \textbf{PHP} : pour la gestion des données côté serveur et l'interaction avec la base de données
\end{itemize}

\section{L'Image Accueil}

\begin{figure}[h]
\centering
\includegraphics[width=1\textwidth]{accueil.png} 
\end{figure}
Notre page d'accueil se distingue par sa simplicité et son design épuré, permettant une navigation intuitive et agréable.
Nous avons choisi un design minimaliste pour notre page d'accueil afin de mettre l'accent sur ce qui compte le plus : les cours. Ainsi, vous pouvez vous concentrer sur votre apprentissage sans être distrait par des éléments superflus.Sur la page d'accueil, vous trouverez le Cours le mieux noté'. Ici, nous mettons en avant le cours qui a reçu les évaluations les plus élevées de la part de nos utilisateurs. C'est un excellent moyen pour vous de découvrir rapidement un contenu de qualité, plébiscité par notre communauté.
\newpage
\section{Connection et Inscription}
\begin{figure}[h]
\centering
\includegraphics[width=1\textwidth]{Formulair.png} 
\end{figure}
Sur Computer Academy, nous avons conçu une page dédiée qui vous permet de vous connecter et de vous inscrire en toute simplicité. Cette page a été pensée pour être facile à utiliser, quel que soit votre niveau d'expérience en ligne.Si vous êtes un nouvel utilisateur, la section d'inscription vous guidera à travers quelques étapes simples pour créer votre compte. Vous aurez besoin de fournir des informations de base telles que votre nom et un mot de passe. Une fois votre compte créé, vous aurez accès à l'ensemble de nos cours et pourrez commencer votre apprentissage.
Si vous avez déjà un compte chez nous, la section de connexion vous permettra de vous connecter rapidement. Il vous suffit d'entrer vos identifiants et votre mot de passe - et vous serez prêt à reprendre là où vous vous étiez arrêté dans vos cours.
\section{La barre de recherche}
\begin{figure}[h]
\centering
\includegraphics[width=0.5\textwidth]{barrederecherche.png} 
\end{figure}
La barre de recherche de Computer Academy a été spécialement conçue pour rendre votre parcours d'apprentissage plus fluide et efficace. Elle ne se contente pas de vous aider à trouver les cours que vous cherchez, mais va plus loin en fournissant des mots-clés pertinents à votre recherche.
En commençant à taper dans la barre de recherche, notre système intelligent commence immédiatement à vous proposer des suggestions en temps réel basées sur les lettres et les mots que vous avez entrés. Ces suggestions peuvent inclure des cours spécifiques, des catégories de cours, ou même des concepts de programmation connexes qui pourraient vous intéresser.
\newpage
\section{Suggetion de cours}
\begin{figure}[h]
\centering
\includegraphics[width=1\textwidth]{suggereruncour.png} 
\includegraphics[width=1\textwidth]{suggereruncour2.png} 
\end{figure}
Si vous avez une idée de cours ou un sujet spécifique que vous aimeriez apprendre, vous pouvez la soumettre pour examen. Pour ce faire, il vous suffit d'aller dans la section 'Suggérer un cours' et de remplir le formulaire avec les détails pertinents de votre suggestion. Nous apprécions vos idées et sommes toujours à la recherche de nouvelles façons d'enrichir notre offre de cours.
Cependant, il est important de noter que toutes les suggestions de cours sont soumises à l'approbation de nos administrateurs. Notre équipe d'administrateurs examine chaque suggestion pour s'assurer qu'elle correspond à notre ligne éditoriale et qu'elle peut apporter une valeur ajoutée à notre communauté d'apprenants. Si votre suggestion est approuvée, nous travaillerons à l'ajout de ce cours à notre plateforme.

\newpage




\section{Mise en page des cours}
\begin{figure}[h]
\centering
\includegraphics[width=0.8\textwidth]{cour.png} 
\end{figure}
Au sommet de chaque page de cours, vous trouverez le titre du cours et un court résumé qui donne une idée générale du contenu et des objectifs d'apprentissage. Ceci est suivi par une évaluation globale du cours donnée par les utilisateurs précédents, vous offrant une indication rapide de la qualité du cours.
Ensuite, le contenu du cours est organisé en modules distincts, chacun se concentrant sur un concept ou une compétence particulière. Ces modules sont présentés dans un ordre logique, facilitant l'apprentissage progressif du sujet.

\section{Espace commentaire}
\begin{figure}[h]
\centering
\includegraphics[width=1\textwidth]{commentaire.png} 
\end{figure}
Nous avons intégré une fonctionnalité de commentaire sur nos pages de cours. Cette fonctionnalité vous permet de partager vos pensées, vos questions, vos réponses, et même vos astuces liées au cours.
Vous pouvez poster un commentaire à la fin de chaque leçon ou module. Que vous souhaitiez discuter d'un concept particulier, poser une question ou simplement partager une idée, nous vous encourageons à utiliser cet espace pour interagir avec d'autres apprenants.
Nous comprenons également que vous pourriez vouloir modifier ou supprimer vos commentaires après les avoir postés. Pour cela, nous avons rendu ce processus simple et rapide. A côté de chaque commentaire que vous avez posté, vous trouverez des options pour le modifier ou le supprimer. Vous pouvez utiliser ces options pour corriger des erreurs, mettre à jour vos pensées, ou retirer un commentaire si vous changez d'avis.
\newpage
\section{Page Profile}

\begin{figure}[h]
\centering
\includegraphics[width=0.95\textwidth]{profile.png} 
\end{figure}
Sur Computer Academy, nous offrons à nos utilisateurs la possibilité de personnaliser leur profil pour une expérience plus personnalisée. Dans la page de profil, vous avez la possibilité de modifier votre nom d'utilisateur, à condition que le nouveau nom d'utilisateur choisi ne soit pas déjà utilisé par un autre membre. C'est une excellente façon de rendre votre profil plus distinct et adapté à votre identité.
Nous comprenons également que dans certains cas, vous pourriez vouloir supprimer votre compte. Nous avons rendu ce processus simple et sans tracas. Dans la section paramètres de votre profil, vous trouverez l'option "Supprimer le compte". Il est important de noter que cette action est irréversible, alors assurez-vous de vouloir supprimer votre compte avant de procéder.
\section{Espace Administrateurs}
\begin{figure}[h]
\centering
\includegraphics[width=1\textwidth]{admine.png} 
\end{figure}
Les administrateurs de notre site ont un contrôle total sur le contenu et les utilisateurs. Ils ont la possibilité de gérer les comptes d'utilisateurs (ajout, modification, suppression), surveiller l'activité des utilisateurs et intervenir en cas de besoin. Ils contrôlent également le contenu du site, pouvant ajouter, modifier ou supprimer des cours. L'accès à l'espace administrateur se fait avec l'identifiant et le mot de passe "admin".

\section{Conception et Planification}
\subsection*{Inspiration}

OpenClassrooms est un site web éducatif qui propose des cours en ligne sur diverses matières, dont la programmation. Son approche pédagogique a grandement influencé la structure et les fonctionnalités de notre site.
NYIP est connu pour son design élégant et professionnel. Nous avons été inspirés par son design pour créer un site visuellement attrayant qui facilite l'apprentissage.
YouTube et Stack Overflow ont également été d'une grande aide. YouTube, avec son immense bibliothèque de tutoriels vidéo, a été une ressource précieuse pour comprendre les concepts de programmation et pour résoudre les problèmes rencontrés lors du développement. Stack Overflow, avec ses nombreuses discussions et solutions à des problèmes de programmation spécifiques, a également été une source d'inspiration pour la conception de notre site et un outil utile pour résoudre les problèmes.
Dans la phase de conception, j'ai d'abord défini les fonctionnalités clés du site : la possibilité pour les utilisateurs de publier, visualiser et supprimer des cours. J'ai ensuite créé des maquettes pour chaque page du site, y compris les pages de cours individuelles et la page principale.

\subsection*{Développement}

Pendant la phase de développement, nous avons construit le squelette du site en HTML, puis stylisé les pages avec CSS. On a utilisé JavaScript pour gérer l'interactivité sur le front-end, comme la navigation entre les pages.
Pour la gestion des cours, nous avons utilisé PHP. Il sert à traiter les demandes des utilisateurs (ajout, visualisation, suppression de cours), où les cours sont stockés.

\subsection*{Test et Déploiement}

Une fois le développement terminé, nous avons effectué une série de tests pour assurer le bon fonctionnement du site. Nous avons vérifié que toutes les fonctionnalités fonctionnaient correctement, y compris l'ajout, la visualisation et la suppression des cours.


\section{Bonus}

Dans le cadre de ce projet, nous avons également inclus quelques fonctionnalités bonus pour améliorer la sécurité et la documentation du site :

\begin{itemize}
\item \textbf{Compte rendu en LaTeX} : Nous avons préparé ce compte rendu en utilisant LaTeX, un système de composition de textes de haute qualité qui est très utilisé pour les documents techniques et scientifiques. Cela a permis de structurer efficacement le rapport et de le présenter de manière professionnelle.

\item \textbf{Mot de passe crypté} : Pour assurer la sécurité des données des utilisateurs, tous les mots de passe sur le site sont cryptés avant d'être stockés dans notre base de données. Cela signifie que même en cas de violation de la sécurité, les mots de passe des utilisateurs resteront protégés.
\end{itemize}

\end{document}